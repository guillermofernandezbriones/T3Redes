\documentclass[12pt]{article}
\usepackage[spanish]{babel}
\usepackage[utf8]{inputenc}
\usepackage[margin=2.5cm]{geometry}
\usepackage{moreverb}
\usepackage[pdftex]{graphicx}
\usepackage{float}
\usepackage{mathtools}
\usepackage{fancyvrb}
\usepackage{color}
\usepackage[math]{iwona}
\usepackage[T1]{fontenc}

\title{\begin{Huge} Redes de Computadores\\
Tarea 3 \\
No somos nada, ¡Hola Internet! (Capa de Red)\end{Huge}}
\author{
\\\\\\\\\textbf{Integrantes}:\\
\begin{tabular}{cc}
	Guillermo Fernández & Álvaro Rojas \\
	guillermo.fernanb.12@alumnos.usm.cl & alvaro.rojasv@alumnos.usm.cl \\
	201073523-0 & 201073555-9 \\
\end{tabular}
\\\\\\\\\textbf{Profesor:}\\
Oscar Encina
\\\\\\ \textbf{Ayudantes:}\\
Alex Arenas\\
Carlos Marchant\\\\\\}
\date{02 de Junio, 2014}

\begin{document}

\maketitle
\newpage
\pagebreak

\section{Introducción}
En este laboratorio, se analizará qué es lo que ocurre con los paquetes a medida que viajan por la gran red de redes que supone Internet, y además, se conocerá cómo funciona el algoritmo vector-distancia con el que los routers completan sus tablas de costos.

\subsection{Objetivos}
\begin{itemize}
\item {\it Evidenciar la real dimensión de la internet y analizar su interconexión.}
\item {\it Conocer el funcionamiento del algoritmo de enrutamiento llamado \emph{Algoritmo de Vector-Distancia}.}
\item {\it Recordar la programación en Python.}
\end{itemize}

\newpage

\section{Desarrollo}
\subsection{Las dimensiones de Internet}
En esta sección, se han analizado una serie de sitios web que, mediante el ingreso de sus URL al programa \textit{Open Visual Traceroute} (OVT), se puede conocer qué camino toman los paquetes a través del planeta, evidenciándose así que existen saltos tan extraordinarios como lo que significa ir de un continente a otro. Los sitios web analizados son los siguientes:

\begin{enumerate}
\item {\it http://moodle.inf.utfsm.cl/}
\item {\it http://google.cl/}
\item {\it http://cime.cl/}
\item {\it http://wikipedia.com/}
\item {\it http://www.chile.embassy.gov.au/}
\end{enumerate}

El \emph{traceroute} realizado para cada sitio web ha sido usando la red del Departamento de Informática de la universidad ("di"). Se debe destacar que al estar conectado a "di", se reconoce al equipo como si estuviera conectado desde Valparaíso, donde se encuentran los servidores del departamento. Otro aspecto a destacar, es que OVT trata como cero los tiempos de latencia y búsqueda de DNS. Si bien en la interfaz gráfica se aprecian como tiempos menores a 1, al copiar la ruta al portapapeles se ilustran como iguales a 0.

\subsubsection{Moodle}
Para \emph{http://moodle.inf.utfsm.cl/}, se tienen la siguiente ruta tomada por los paquetes y los tiempos que han demorado aquellos paquetes en cada router a través del globo:\\

\begin{table}[H]
\centering
\begin{tabular}{| c | c | c | c | c | c |}
\hline
\# & País & Localidad & Latitud & Longitud & IP\\
\hline
1 & Chile & Valparaiso & -33.047806 & -71.6011 & 200.1.20.155\\
\hline
2 & Chile & Valparaiso & -33.047806 & -71.6011 & 200.1.21.155\\
\hline
3 & Chile & Valparaiso & -33.047806 & -71.6011 & 200.1.19.40\\
\hline
\end{tabular}
\caption{\small \textbf{Ruta por los paquetes al consultar la página de Moodle.}}
\end{table}

\begin{table}[H]
\centering
\begin{tabular}{| c | c | c | c | c |}
\hline
\# & Hostname & Latencia & Búsqueda & Distancia al nodo\\
 &  & (ms) & DNS (ms) & anterior (km)\\
\hline
1 & nat-sj-labinf1.campus.utfsm.cl & 0 & 0 & 0\\
\hline
2 & gw-ext.inf.utfsm.cl. & 160 & 3 & 0\\
\hline
3 & moodle.inf.utfsm.cl. & 140 & 4 & 0\\
\hline
\end{tabular}
\caption{\small \textbf{Continuación de la tabla anterior.}}
\end{table}

\begin{figure}[H] 
\centering 
\includegraphics[width=1\textwidth]{imagenes/moodle_grafica.png} \caption{\small \textbf{Tiempos de latencia de paquetes al ingresar a Moodle.}} 
\label{fig:diagrama_1} 
\end{figure}

\subsubsection{Google}
La tabla y gráficas siguientes corresponden a \emph{http://google.com/}:\\

\begin{table}[H]
\centering
\begin{tabular}{| c | c | c | c | c | c |}
\hline
\# & País & Localidad & Latitud & Longitud & IP\\
\hline
1 & Chile & Valparaiso & -33.047806 & -71.6011 & 200.1.20.155\\
\hline
2 & United States & Chicago & 41.884903 & -87.6238 & 192.169.2.1\\
\hline
3 & United States & Chicago & 41.884903 & -87.6238 & 192.169.3.2\\
\hline
4 & Chile & Valparaiso & -33.047806 & -71.6011 & 200.1.21.131\\
\hline
5 & Chile & Valparaiso & -33.047806 & -71.6011 & 200.1.20.131\\
\hline
6 & Chile & Santiago & -33.449997 & -70.6667 & 190.208.0.173\\
\hline
7 & Chile & Santiago & -33.449997 & -70.6667 & 190.208.5.57\\
\hline
8 & Chile & Santiago & -33.449997 & -70.6667 & 200.27.5.186\\
\hline
9 & Chile & Santiago & -33.449997 & -70.6667 & 190.208.5.86\\
\hline
10 & Chile & Santiago & -33.449997 & -70.6667 & 190.208.9.13\\
\hline
11 & Chile & Santiago & -33.449997 & -70.6667 & 190.208.9.26\\
\hline
12 & United States & Mountain View & 37.419205 & -122.0574 & 72.14.234.41\\
\hline
13 & United States & Mountain View & 37.419205 & -122.0574 & 173.194.42.216\\
\hline
\end{tabular}
\caption{\small \textbf{Ruta por los paquetes al consultar la página de Google.}}
\end{table}

\begin{table}[H]
\centering
\begin{tabular}{| c | c | c | c | c |}
\hline
\# & Hostname & Latencia & Búsqueda & Distancia al nodo\\
 &  & (ms) & DNS (ms) & anterior (km)\\
\hline
1 & nat-sj-labinf1.campus.utfsm.cl & 0 & 0 & 0\\
\hline
2 & (None) & 1 & 2 & 8501\\
\hline
3 & (None) & 3 & 2 & 0\\
\hline
4 & fw.usm.cl. & 2 & 7 & 8501\\
\hline
5 & telmex-gw.usm.cl. & 2 & 3 & 0\\
\hline
6 & 190.208.0.173. & 4 & 9 & 97\\
\hline
7 & 190.208.5.57. & 2 & 17 & 0\\
\hline
8 & (None) & 1 & 3 & 0\\
\hline
9 & 190.208.5.86. & 1 & 13 & 0\\
\hline
10 & 190.208.9.13. & 1 & 2 & 0\\
\hline
11 & 190.208.9.26. & 1 & 2 & 0\\
\hline
12 & (None) & 9 & 153 & 9517\\
\hline
13 &  scl03s05-in-f24.1e100.net. & 2 & 3 & 0\\
\hline
\end{tabular}
\caption{\small \textbf{Continuación de la tabla anterior.}}
\end{table}

\begin{figure}[H] 
\centering 
\includegraphics[width=1\textwidth]{imagenes/google_grafica.png} \caption{\small \textbf{Tiempos de latencia de paquetes al ingresar a Google.}} 
\label{fig:diagrama_1} 
\end{figure}

\subsubsection{CIME}
La tabla y gráficas que ahora siguen son las que corresponden al ingreso de \emph{http://cime.cl/}:\\

\begin{table}[H]
\centering
\begin{tabular}{| c | c | c | c | c | c |}
\hline
\# & País & Localidad & Latitud & Longitud & IP\\
\hline
1 & Chile & Valparaiso & -33.047806 & -71.6011 & 200.1.20.155\\
\hline
2 & Chile & Valparaiso & -33.047806 & -71.6011 & 200.1.21.131\\
\hline
3 & Chile & Valparaiso & -33.047806 & -71.6011 & 200.1.20.155\\
\hline
4 & Chile & Santiago & -33.449997 & -70.6667 & 190.208.0.173\\
\hline
5 & Chile & Santiago & -33.449997 & -70.6667 & 190.208.5.77\\
\hline
6 & Chile & Santiago & -33.449997 & -70.6667 & 200.27.5.178\\
\hline
7 & Chile & Santiago & -33.449997 & -70.6667 & 200.27.5.114\\
\hline
8 & Chile & Santiago & -33.449997 & -70.6667 & 190.208.9.9\\
\hline
9 & Chile & Santiago & -33.449997 & -70.6667 & 190.208.9.6\\
\hline
10 & United States & Englewood & 39.623703 & -104.8738 & 129.250.2.110\\
\hline
11 & United States & Englewood & 39.623703 & -104.8738 & 129.250.2.99\\
\hline
12 & United States & Englewood & 39.623703 & -104.8738 & 129.250.4.4\\
\hline
13 & United States & New York & 40.7267 & -73.9981 & 192.241.164.238\\
\hline
14 & United States & New York & 40.7267 & -73.9981 & 107.170.72.180\\
\hline
\end{tabular}
\caption{\small \textbf{Ruta por los paquetes al consultar la página de CIME.}}
\end{table}

\begin{table}[H]
\centering
\begin{tabular}{| c | c | c | c | c |}
\hline
\# & Hostname & Latencia & Búsqueda & Distancia al nodo\\
 &  & (ms) & DNS (ms) & anterior (km)\\
\hline
1 & nat-sj-labinf1.campus.utfsm.cl & 0 & 0 & 0\\
\hline
2 & fw.usm.cl. & 37 & 3 & 0\\
\hline
3 & telmex-gw.usm.cl. & 27 & 3 & 0\\
\hline
4 & 190.208.0.173. & 1 & 186 & 97\\
\hline
5 & 190.208.5.77. & 0 & 46 & 0\\
\hline
6 & (None) & 1 & 289 & 0\\
\hline
7 & Ge1-2.igr1.Santiago.ip.telmexchile.cl. & 59 & 84 & 0\\
\hline
8 & 190.208.9.9. & 36 & 62 & 0\\
\hline
9 & 190.208.9.6. & 36 & 27 & 0\\
\hline
10 & ae-3.r20.miamifl02.us.bb.gin.ntt.net. & 1 & 4 & 8864\\
\hline
11 & ae-8.r21.asbnva02.us.bb.gin.ntt.net. & 10 & 3 & 0\\
\hline
12 & ae-8.r20.asbnva02.us.bb.gin.ntt.net. & 46 & 4 & 0\\
\hline
13 & (None) & 34 & 10507 & 2615\\
\hline
14 & (None) & 1 & 369 & 0\\
\hline
\end{tabular}
\caption{\small \textbf{Continuación de la tabla anterior.}}
\end{table}

\begin{figure}[H] 
\centering 
\includegraphics[width=1\textwidth]{imagenes/cime_grafica.png} \caption{\small \textbf{Tiempos de latencia de paquetes al ingresar a CIME.}}
\label{fig:diagrama_1} 
\end{figure}

\subsubsection{Wikipedia}
Las tablas y gráfica para el ingreso a \emph{http://wikipedia.com/} son las siguientes:\\

\begin{table}[H]
\centering
\begin{tabular}{| c | c | c | c | c | c |}
\hline
\# & País & Localidad & Latitud & Longitud & IP\\
\hline
1 & Chile & Valparaiso & -33.047806 & -71.6011 & 200.1.20.155\\
\hline
2 & Chile & Valparaiso & -33.047806 & -71.6011 & 200.1.21.131\\
\hline
3 & Chile & Valparaiso & -33.047806 & -71.6011 & 200.1.20.131\\
\hline
4 & Chile & Santiago & -33.449997 & -70.6667 & 190.208.0.173\\
\hline
5 & Chile & Santiago & -33.449997 & -70.6667 & 190.208.5.77\\
\hline
6 & Chile & Santiago & -33.449997 & -70.6667 & 200.27.5.178\\
\hline
7 & Chile & Santiago & -33.449997 & -70.6667 & 200.27.5.206\\
\hline
8 & Chile & Santiago & -33.449997 & -70.6667 & 190.208.9.9\\
\hline
9 & Chile & Santiago & -33.449997 & -70.6667 & 190.208.9.6\\
\hline
10 & United States & Englewood & 39.623703 & -104.8738 & 157.238.179.17\\
\hline
11 & United States & Englewood & 39.623703 & -104.8738 & 129.250.2.184\\
\hline
12 & United States & Englewood & 39.623703 & -104.8738 & 129.250.4.207\\
\hline
13 & United States & Englewood & 39.623703 & -104.8738 & 129.250.204.190\\
\hline
14 & United States & San Francisco & 37.789795 & -122.394196 & 208.80.154.224\\
\hline
\end{tabular}
\caption{\small \textbf{Ruta por los paquetes al consultar la página de Wikipedia.}}
\end{table}

\begin{table}[H]
\centering
\begin{tabular}{| c | c | c | c | c |}
\hline
\# & Hostname & Latencia & Búsqueda & Distancia al nodo\\
 &  & (ms) & DNS (ms) & anterior (km)\\
\hline
1 & nat-sj-labinf1.campus.utfsm.cl & 0 & 0 & 0\\
\hline
2 & fw.usm.cl. & 1 & 11 & 0\\
\hline
3 & telmex-gw.usm.cl. & 1 & 3 & 0\\
\hline
4 & 190.208.0.173. & 7 & 5 & 97\\
\hline
5 & 190.208.5.77. & 2 & 4 & 0\\
\hline
6 & (None) & 8 & 2 & 0\\
\hline
7 & (None) & 105 & 16 & 0\\
\hline
8 & 190.208.9.9. & 2 & 5 & 0\\
\hline
9 & 190.208.9.6. & 1 & 11 & 0\\
\hline
10 & xe-0-4-0-2.r05.miamfl02.us.bb.gin.ntt.net. & 69 & 6 & 8864\\
\hline
11 & ae-4.r20.miamfl02.us.bb.gin.ntt.net. & 25 & 3 & 0\\
\hline
12 & ae-2.r04.asbnva02.us.bb.gin.ntt.net. & 140 & 22 & 0\\
\hline
13 & xe-0-7-0-8.r04.asbnva02.us.ce.gin.ntt.net. & 69 & 3 & 0\\
\hline
14 & text-lb.eqiad.wikimedia.org. & 78 & 1584 & 1533\\
\hline
\end{tabular}
\caption{\small \textbf{Continuación de la tabla anterior.}}
\end{table}

\begin{figure}[H] 
\centering 
\includegraphics[width=1\textwidth]{imagenes/wiki_grafica.png} \caption{\small \textbf{Tiempos de latencia de paquetes al ingresar a Wikipedia.}}
\label{fig:diagrama_1} 
\end{figure}

\subsubsection{Embajada de Australia en Chile}
Al ingresar a \emph{http://www.chile.embassy.gov.au/} se obtuvieron las siguientes tablas y gráfica que representa latencias. Cabe destacar que para el traceroute realizado a este sitio web, al llegar al router situado en Japón, OVT se quedó colgado, así que la ejecución terminó en ese punto.\\

\begin{table}[H]
\centering
\begin{tabular}{| c | c | c | c | c | c |}
\hline
\# & País & Localidad & Latitud & Longitud & IP\\
\hline
1 & Chile & Valparaiso & -33.047806 & -71.6011 & 200.1.20.155\\
\hline
2 & United States & Chicago & 41.884903 & -87.6238 & 192.169.2.1\\
\hline
3 & United States & Chicago & 41.884903 & -87.6238 & 192.169.3.2\\
\hline
4 & Chile & Valparaiso & -33.047806 & -71.6011 & 200.1.21.131\\
\hline
5 & Chile & Valparaiso & -33.047806 & -71.6011 & 200.1.20.131\\
\hline
6 & Chile & Santiago & -33.449997 & -70.6667 & 190.208.0.173\\
\hline
7 & Chile & Santiago & -33.449997 & -70.6667 & 190.208.5.53\\
\hline
8 & Chile & Santiago & -33.449997 & -70.6667 & 200.27.5.178\\
\hline
9 & Chile & Santiago & -33.449997 & -70.6667 & 200.27.5.114\\
\hline
10 & Chile & Santiago & -33.449997 & -70.6667 & 190.208.9.9\\
\hline
11 & Chile & Santiago & -33.449997 & -70.6667 & 190.208.9.86\\
\hline
12 & United States & New York & 40.758804 & -73.968 & 173.241.129.209\\
\hline
13 & France & (Unknown) & 48.86 & 2.350006 & 77.67.68.234\\
\hline
14 & France & (Unknown) & 48.86 & 2.350006 & 141.136.105.34\\
\hline
15 & Japan & (Unknown) & 35.690002 & 139.69 & 202.147.50.186\\
\hline
16 & Asia/Pacific Region & (Unknown) & 35.0 & 105.0 & 203.192.174.137\\
\hline
17 & Asia/Pacific Region & (Unknown) & 35.0 & 105.0 & 203.192.174.165\\
\hline
18 & Japan & (Unknown) & 35.690002 & 139.69 & 202.147.42.160\\
\hline
\end{tabular}
\caption{\small \textbf{Ruta por los paquetes al consultar la página de la embajada de Australia en Chile.}}
\end{table}

\begin{table}[H]
\centering
\begin{tabular}{| c | c | c | c | c |}
\hline
\# & Hostname & Latencia & Búsqueda & Distancia al nodo\\
 &  & (ms) & DNS (ms) & anterior (km)\\
\hline
1 & nat-sj-labinf1.campus.utfsm.cl & 0 & 0 & 0\\
\hline
2 & (None) & 2 & 7 & 8501\\
\hline
3 & (None) & 3 & 2 & 0\\
\hline
4 & fw.usm.cl. & 42 & 30 & 8501\\
\hline
5 & telmex-gw.usm.cl. & 2 & 5 & 0\\
\hline
6 & 190.208.0.173. & 41 & 3 & 97\\
\hline
7 & 190.208.5.53. & 1 & 4 & 0\\
\hline
8 & (None) & 7 & 7 & 0\\
\hline
9 & Ge1-2.igr1.Santiago.ip.telmexchile.cl. & 25 & 2 & 0\\
\hline
10 & 190.208.9.9. & 0 & 3 & 0\\
\hline
11 & 190.208.9.86. & 1 & 3 & 0\\
\hline
12 & ae2-202.nyc20.ip4.tinet.net. & 140 & 5 & 8267\\
\hline
13 & pacnet-gw.ip4.tinet.net. & 56 & 4 & 5837\\
\hline
14 & (None) & 10 & 12 & 0\\
\hline
15 & be1.gw4.sjc1.asianetcom.net. & 12 & 3 & 9722\\
\hline
16 & te0-1-0-0-981.cr2.syd5.asianetcom.net. & 151 & 22 & 3134\\
\hline
17 & te0-0-0-0.cr1.syd5.asianetcom.net. & 1 & 8 & 0\\
\hline
18 & ge-2-1-0-0.gw1.cbr1.asianetcom.net. & 129 & 4 & 3134\\
\hline
\end{tabular}
\caption{\small \textbf{Continuación de la tabla anterior.}}
\end{table}

\begin{figure}[H] 
\centering 
\includegraphics[width=1\textwidth]{imagenes/australia_grafica_sinfin.png} \caption{\small \textbf{Tiempos de latencia de paquetes al ingresar al sitio de la embajada de Australia en Chile.}}
\label{fig:diagrama_1} 
\end{figure}

\newpage

\subsection{Aplicación del Algoritmo de Vector-Distancia}

\end{document}